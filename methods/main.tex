\documentclass[11pt]{article}
\usepackage[margin=1in]{geometry}
\usepackage{fancyvrb}
\usepackage{natbib}
\bibliographystyle{unsrtnat}
\usepackage{amsmath,amssymb}
\DeclareMathOperator{\se}{\textrm{se}}
\title{Supplementary Methods}
\author{Anqi Zhu$^\dagger$, Nana Matoba$^\dagger$, Emmaleigh Wilson, Amanda L. Tapia,
  \\ Yun Li, Joseph G. Ibrahim, Jason L. Stein, Michael I. Love}
\begin{document}
\maketitle
\section{MRLocus statistical model}

MRLocus proceeds in two separate hierarchical models, which are
encoded in the Stan programming language and with posterior
inference performed using the Stan and RStan software packages
\citep{stan,rstan}. In section \ref{sec:coloc} we define the
model for the colocalization step, and in section \ref{sec:slope} we
define the model for the slope fitting step.

\subsection{Colocalization step} \label{sec:coloc}

\subsubsection{Input data}

The first step performs colocalization of eQTL (A) and GWAS (B)
signals across a number of LD-independent signal clusters
$j \in 1,\dots,J$, using summary statistics from both studies:
$\widehat{\beta}^A_{i,j}$ and $\se(\widehat{\beta}^A_{i,j})$,
$\widehat{\beta}^B_{i,j}$ and $\se(\widehat{\beta}^B_{i,j})$
for SNP $i \in 1,\dots,n_j$ in cluster $j$ for study A and B,
and the respective LD matrices for each cluster $j$:
$\Sigma_j^A$ and $\Sigma_j^B$.
The clusters and the individual SNPs per cluster must be matched across
study, and our procedure for performing this matching is described in
the Methods section in the main text.
The estimated coefficients $\widehat{\beta}^X_{i,j}$ for
$X \in \{A,B\}$ refer to either the estimated coefficients from a
linear model of a continuous trait $y$ on genotype dosages
$\{0,1,2\}$, or the estimated log odds from a logistic regression of a
binary trait $y$ modeled on genotype dosages.

While it would be preferable to use allelic fold change (aFC)
\citep{aFC} or ACME effect sizes \citep{ACME} for the eQTL (A) study
in MRLocus modeling, in practice we typically are provided with
publicly available estimated coefficients representing inverse normal
transformed (INT) or $\log_2$ transformed expression values regressed
on genotype dosages. For eQTL coefficients derived from INT expression
data, the mediation effect estimated by MRLocus represents the effect
on the trait from modifying gene expression by 1 SD, while for eQTL
coefficients derived from $\log_2$ transformed expression data, the
mediation effect represents the effect on the trait from doubling gene
expression.

The eQTL and GWAS studies are referred to as ``A'' and ``B'' in the
formula and code below for generalization, for example, the eQTL study
could be replaced with a pQTL (protein quantitative trait loci)
study. ``A'' therefore refers to a study of a trait (or ``exposure'')
that is believed to be causally upstream of the trait examined in
study ``B'' (or ``outcome'').

\subsubsection{Collapsing and allele flipping}

MRLocus contains two convenience functions,
\texttt{collapseHighCorSNPs} and \texttt{flipAllelesAndGather}, which
are described briefly. The first function uses hierarchical clustering
based on the LD matrix (the user must pick which to use if two are
available), in order to collapse SNPs into groups using complete
linkage, and thresholding the resulting dendrogram at 0.95
correlation. The SNP with the highest absolute $z$-score within a
collapsed set is chosen as the representative SNP.

\texttt{flipAllelesAndGather} performs a number of allele flipping
steps for assisting statistical modeling and visualization. The
alleles are flipped such that the index SNP (as defined by its
absolute value of $z$-score) for study A has a positive estimated
coefficient. This is to simplify the interpretations of the plots --
such that we are always describing the effects on downstream traits
for expression increasing alleles. Additionally, we flip alleles such
that SNPs with positive correlation of genotypes in either
$\Sigma_j^A$ or $\Sigma_j^B$ (the user must pick which to use) are
kept the same, while SNPs with negative correlation of genotypes have
their alleles flipped. Allele flipping involves both keeping track of
the reference and effect allele, as well as changing the sign of the
estimated coefficient. This function also performs checks such that
the A and B study agree in terms of the effect and reference allele.
If two LD matrices are provided, one is prioritized for generating
positive correlations of genotypes, while the other has its alleles
flipped for consistency.

\subsubsection{Scaling}

In practice, before supplying
$\widehat{\beta}^A_{i,j}$ and $\widehat{\beta}^B_{i,j}$
and the associated standard errors to the colocalization hierarchical
model, the values are scaled such that the index SNP (as defined by
its absolute value of $z$-score) for study A has estimated coefficient
of $\pm 1$ for both study A and B. If two or more SNPs have the same
$z$-score, the first is chosen.  This simplifies the Stan code and
improves model fit, as the two studies are then at comparable
scale. The scaling is reversed after the Stan model is fit.  For the
user-input estimated coefficients and standard errors for
study $X \in \{A,B\}$ and cluster $j$,
$\widehat{\beta}^{X,input}_{i,j}$ and $\se(\widehat{\beta}^{X,input}_{i,j})$,
in the first step we create scaled estimated coefficients:

\begin{align}
  z^*_j &= \max_i \left( |\widehat{\beta}^{A,input}_{i,j}|/\se(\widehat{\beta}^{A,input}_{i,j}) \right) \\
  i^*_j &= \min \left(i \in 1,\dots,n_j \right) \  s.t. \  |\widehat{\beta}^{A,input}_{i,j}|/\se(\widehat{\beta}^{A,input}_{i,j}) = z^*_j \\
  s^X_j &= 1/|\widehat{\beta}^{X,input}_{i^*,j}| \\
  \widehat{\beta}^X_{i,j} &= s_j^X \widehat{\beta}^{X,input}_{i,j}, \; i \in 1,\dots,n_j \\
  \se(\widehat{\beta}^X_{i,j}) &= s_j^X \se(\widehat{\beta}^{X,input}_{i,j}), \; i \in 1,\dots,n_j
\end{align}

Note that equations (1-2) refer specifically to study A, while
equations (3-5) refer to steps that are repeated for $X=A$ and
$X=B$. Again, in words, $i^*_j$ is the index of the first occurrence of
the maximal value of absolute value of $z$-score in study A and cluster
$j$.

\subsubsection{Colocalization}

Colocalization refers to the task of determining if the same signal in
eQTL and GWAS summary statistics arise from the same causal variant(s),
considering the correlation of genotypes in a locus (the LD
matrix). Here we perform colocalization using a generative model for
the estimated coefficients, where the true coefficients will be
modeled and their posterior distribution used for inference.
In the following equations, $\widehat{\beta}^X_{i,j}$ and
$\se(\widehat{\beta}^X_{i,j})$ refer to these scaled estimates as
described in the previous section, but in the following section
\ref{sec:slope} on slope fitting, the values are transformed back to
the original scale by multiplying by $1/s^X_j$ for $X \in \{A,B\}$
respectively.

We use a statistical model for the summary statistics motivated by the
eCAVIAR model \citep{eCAVIAR}. In eCAVIAR, the summary statistic
$z$-scores in a vector $S$ are modeled as a multivariate normal
distribution with mean vector $\Sigma \Lambda$ and covariance matrix
$\Sigma$ (the LD matrix), where $\Lambda$ is a vector giving the true
standardized effect sizes. In a locus with a single causal SNP
producing the observed enrichment of signal, and if the causal SNP is
in the set modeled by eCAVIAR, $\Lambda$ would consist of a vector
with all 0 values except for one SNP with non-zero value.  eCAVIAR
then models $\Lambda$ with a multivariate normal prior distribution
centered on 0 with covariance matrix based on a vector of 0's and 1's
giving the true causal status of the SNPs in the locus and a preset
scale parameter determined from previous studies.

Here MRLocus diverges from eCAVIAR in two ways. First, we will use a
generative model for the elements in the vector
$\widehat{\beta}^X_{\centerdot,j}$ conditional on the true effect sizes
$\beta^X_{\centerdot,j}$ within LD-independent signal cluster
$j$ (that is, MRLocus models effect sizes rather than $z$-scores).
Second, we will use a univariate distribution for each
$\widehat{\beta}^X_{i,j}$ instead of a multivariate distribution for
the entire vector. This second difference is motivated by practical
concerns of model fitting and specification; the univariate modeling
provided more efficient model fitting with Stan (higher effective
sample sizes and R-hat values near 1), and allowed for more
flexible choice of priors, as described below. 

Instead of a multivariate normal prior on the estimated coefficients,
MRLocus uses a horseshoe prior \citep{horseshoe1,horseshoe2}, 
a type of hierarchical shrinkage prior that provides sparsity in the
posterior estimates \citep{hiershrink}. In the following, we use the
notation of \citet{horseshoe1}, where $\lambda_i$ provides the
\emph{local} shrinkage parameters and $\tau$ provides the
\emph{global} shrinkage parameter.

The following hierarchical model is fit separately across
LD-independent signal clusters $j \in 1,\dots,J$, and so in the
following equations, the subscript for $j$ is omitted for clarity. In
all equations below but the last, $i \in 1,\dots,n_j$. Here, for
consistency with Stan code, the normal distribution is written as
$N(\mu,\sigma)$ where the second element $\sigma$ provides the
standard deviation instead of the variance.

\begin{align}
  \widehat{\beta}^A_i &\sim N([\Sigma^A \beta^A]_i, \se(\widehat{\beta}^A_i)) \\
  \widehat{\beta}^B_i &\sim N([\Sigma^B \beta^B]_i, \se(\widehat{\beta}^B_i)) \\
  \beta_i^A &\sim N(0, \lambda_i \tau) \\
  \beta_i^B &\sim N(0, \lambda_i \tau) \\
  \lambda_i &\sim \textrm{Cauchy}(0,1) \\
  \tau &\sim \textrm{Cauchy}(0,1)
\end{align}

Of note, $\beta_i^A$ and $\beta_i^B$ share a prior involving
$\lambda_i$, such that evidence from study A and B supporting a
SNP $i^\dagger$ that is causal both for eQTL and GWAS signal
($\beta_{i^\dagger}^A \ne 0, \beta_{i^\dagger}^B \ne 0$),
will be reflected in larger posterior draws for $\lambda_{i^\dagger}$
compared to other $\lambda_{i'}$ for $i' \ne i^\dagger$.

The posterior mean for $\beta_{i,j}^A$ and $\beta_{i,j}^B$ are the
parameters of interest from this step of model fitting, and passed
along to the next step after scaling back to the original scale by
$1/s_j^A$ and $1/s_j^B$, respectively, as described earlier.
We also considered making use of the posterior standard deviation
for these two parameters, but found MRLocus gave better performance in
terms of accuracy and stability in the Stan fitting procedure if only
the posterior mean was kept from the colocalization step. 
The use of the horseshoe prior in this colocalization step
is distinct from other uses of the horseshoe prior in Bayesian
Mendelian Randomization methods, \citet{Berzuini2018} and
\citet{Uche2019}, where it is used as a prior for the
pleiotropic effects (effects not mediated by the exposure) or on the
mediation slope, respectively.

\subsection{Slope fitting step} \label{sec:slope}

The second step of MRLocus is to estimate the gene-to-trait effect
(the slope in a Mendelian Randomization analysis) using the
posterior mean values from the colocalization step. For each
LD-independent signal cluster $j$, MRLocus extracts one SNP, based on the
largest posterior mean value for $\beta_{i,j}^A$ (prioritizing the
putative mediator for SNP selection). MRLocus also has the
non-default option to perform model-based clustering using 
an EM algorithm to determine how many SNPs per LD-independent
signal cluster to pass to the slope fitting step \citep{mclust}, but
this option was not extensively evaluated here.

By default, $J$ SNPs are passed from the colocalization step to the
slope fitting step, which should represent the candidate causal SNPs
from each LD-independent signal cluster. The posterior mean for
$\beta_{i,j}^A$ and $\beta_{i,j}^B$ for these SNPs is then provided to
the slope fitting step as variables $\widehat{\beta}_j^A$ and
$\widehat{\beta}_j^B$. These two variables
are modeled as normally distributed variables centered on
true values $\beta_j^A$ and $\beta_j^B$
with standard deviation according to the original standard errors
$\se(\widehat{\beta}_{i,j}^A)$ and
$\se(\widehat{\beta}_{i,j}^B)$. We found this incorporation of the
original standard errors into the slope fitting procedure helped to
accurately estimate the uncertainty on the slope (the gene-to-trait
effect). 

The primary two parameters of interest are the slope $\alpha$, 
i.e. the predominant gene-to-trait effect demonstrated by the
LD-independent signal clusters in the locus, and $\sigma$,
the dispersion of gene-to-trait effects from individual signal clusters
around the predominant slope.
The hierarchical model for the slope fitting step is given by:

\begin{align}
  \widehat{\beta}_j^A &\sim N(\beta_j^A, \se_j^A) \\
  \widehat{\beta}_j^B &\sim N(\beta_j^B, \se_j^B) \\
  \beta_j^A &\sim N(0, \textrm{SD}_\beta) \\
  \beta_j^B &\sim N(\alpha \beta_j^A, \sigma) \\
  \alpha &\sim N(\mu_\alpha, \textrm{SD}_\alpha) \\
  \sigma &\sim HN(0, \textrm{SD}_\sigma)
\end{align}

where the first four equations are defined for $j \in 1,\dots,J$,
and where $HN$ specifies the Half-Normal distribution.
In words, $\beta_j^B$ is assumed to follow a normal distribution
centered on the predicted value from a line with slope $\alpha$ and no
intercept, and with dispersion $\sigma$ around the fitted line.
We are both interested in the posterior mean of $\alpha$ and $\sigma$
as well as our uncertainty regarding the point estimates. In
particular we focus on a quantile-based credible interval for
$\alpha$, which is used for determining our confidence in a gene being
a causal mediator for a trait. Large values of $\sigma$ relative to
$\alpha$ times typical mediator perturbation sizes ($\beta_j^A$)
reflect significant heterogeneity of the gene-to-trait effect.

\subsubsection{Choice of hyperparameters}

The hyperparameter for the prior for $\beta_j^A$ is
$\textrm{SD}_\beta$, which is set to 2 times the largest absolute value
of $\widehat{\beta}_j^A$.
The hyperparameters for the prior for $\alpha$ are
$\mu_\alpha$ and $\textrm{SD}_\alpha$, and 
are set based on a simple un-weighted linear model of
$\widehat{\beta}_j^B$ on $\widehat{\beta}_j^A$ without an intercept.
$\mu_\alpha$ is set to the estimated slope coefficient, and
$\textrm{SD}_\alpha$ is set to 2 times the absolute value of this
estimated slope coefficient.

The hyperparameter for the prior for $\sigma$ is
$\textrm{SD}_\sigma$, which is set by default to
1 (but can be modified by the user). This corresponds to a very wide
prior as the GWAS effects sizes are typically quite small in magnitude
(as seen in the analyses in the Results).

\subsubsection{One signal cluster}

In the case that the upstream clumping method only returns a single
LD-independent signal cluster, MRLocus relies on a parametric
bootstrap to estimate the slope. MRLocus samples the numerator and
denominator of the slope from normal distributions 
$N(\widehat{\beta}_1^B,\se_1^B)$ and $N(\widehat{\beta}_1^A, \se_1^A)$,
using the posterior mean estimates from section \ref{sec:coloc}.
The point estimate is then the median of the bootstrap slope
estimates. Intervals are calculated using the MAD of the bootstrap
slope estimates, e.g. the median $\pm 1.28$ times the MAD to construct
an 80\% interval.

\newpage

\section{MRLocus Stan code}

\subsection{Colocalization step}

\texttt{inst/stan/beta\_coloc.stan}

\begin{Verbatim}[frame=single,numbers=left]
data {
  int n;
  vector[n] beta_hat_a;
  vector[n] beta_hat_b;
  vector[n] se_a;
  vector[n] se_b;
  matrix[n,n] Sigma_a;
  matrix[n,n] Sigma_b;
}
parameters {
  vector[n] beta_a;
  vector[n] beta_b;
  vector<lower=0>[n] lambda;
  real<lower=0> tau;
}
model {
  tau ~ cauchy(0, 1);
  lambda ~ cauchy(0, 1);
  beta_hat_a ~ normal(Sigma_a * beta_a, se_a);
  beta_hat_b ~ normal(Sigma_b * beta_b, se_b);
  for (i in 1:n) {
    beta_a[i] ~ normal(0, lambda[i] * tau);
    beta_b[i] ~ normal(0, lambda[i] * tau);
  }
}
\end{Verbatim}

\newpage

\subsection{Slope fitting step}

\texttt{inst/stan/slope.stan}

\begin{Verbatim}[frame=single,numbers=left]
data {
  int n; 
  vector[n] beta_hat_a;
  vector[n] beta_hat_b;
  vector[n] sd_a;
  vector[n] sd_b;
  real sd_beta;
  real mu_alpha;
  real sd_alpha;
  real sd_sigma;
}
parameters {
  real alpha;
  real<lower=0> sigma;
  vector[n] beta_a;
  vector[n] beta_b;
}
model {
  beta_hat_a ~ normal(beta_a, sd_a);
  beta_hat_b ~ normal(beta_b, sd_b);
  beta_a ~ normal(0, sd_beta);
  beta_b ~ normal(alpha * beta_a, sigma);
  alpha ~ normal(mu_alpha, sd_alpha);
  sigma ~ normal(0, sd_sigma);
}
\end{Verbatim}

\newpage

\bibliography{main}
\end{document}
